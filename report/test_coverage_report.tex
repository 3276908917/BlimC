\documentclass[12pt]{article}

\usepackage[margin=1in]{geometry}

\usepackage{amsmath}
\usepackage{graphicx}

\usepackage{caption, subcaption, xcolor}

\graphicspath{{res/}}

\title{Lab 2}
\author{Lukas Finkbeiner}

\begin{document}

\maketitle

\begin{abstract}

We investigate the properties of the signal from the hyperfine hydrogen transition with a collector dish connected to an ADC by several layers of mixers and filters, so that we are left with frequencies more amenable to currently-realizable sampling rates. We calibrate our instruments to the thermal noise of our system and by averaging over many blocks of consecutive data. We use humans, as studied blackbodies, to demonstrate the principle behind thermal calibration. We also include rotation matrices and our timestamps for some of the data that we collected, as an illustration of the Doppler corrections to be applied when converting frequencies to Doppler velocities on a power spectrum. We offer examples of all these processes with a measurement of the HI line at zenith and when pointing at Cassiopeia. In a second experiment, we investigate the speed of light by measuring minima in the magnitude-squared voltages of standing waves in a waveguide. While we lay out the theory and process behind relevant data acquisition for the 3GHz and XBand waveguides, the least-squares chi-squared approach to fitting the nonlinear XBand formula was not completed successfully. Results are offered for the 3GHz waveguide, but without uncertainty on the parameters. We relate the voltage standing wave ratio and null position offset with ideas regarding transmission and impedance matching.

\end{abstract}

\section{Introduction and Background}

\quad \quad In the first of two experiments, we seek to observe the 21-cm HI line, at 1420.4 MHz. However, we want to look at frequencies better suited to our lab equipment. For example, our PicoScope has a maximum sampling rate of 62.5 MHz, so anything above 125 MHz would be aliased.

The data that we sample from the Big Horn via the PicoScope will feature arbitrary units which depend on our setup. To calibrate the intensity of the spectrum, we first calculate the gain:

\begin{equation} \label{eq:thermal}
G = \frac{T_\text{sys, cal} - T_\text{sys, cold}}{\sum (s_\text{cal} - s_\text{cold})} \sum s_\text{cold} \approx \frac{\text{300 K}}{\sum (s_\text{cal} - s_\text{cold})} \sum s_\text{cold}
\end{equation}

Where $T_\text{sys, cal} \approx 98.6^\circ$ F $\approx 310$ K represents the temperature of the human flesh that we use to cover most of the telescope. $T_\text{sys, cold}$ represents the temperature of the cold sky. We expect $T_\text{sys, cold} \ll T_\text{sys, cal}$, whereby we get the approximated form on the right. $s_\text{cal}$ represents the spectrum for which we attempt to maximize thermal noise in the collector, and $s_\text{cold}$ represents the spectrum for which we attempt to minimize.

To remove constant sources of interference in our data and to obtain the shape of the line, we divide an `on' spectrum by an `off' spectrum. 

\begin{equation}
s_\text{line} = \frac{s_\text{on}}{s_\text{off}}
\end{equation}

The fully-calibrated spectrum represents a scaling of this $s_\text{line}$ by the thermal gain from equation \ref{eq:thermal}:

\begin{equation} \label{eq:normalize}
T_\text{line} = s_{line} \times G
\end{equation}

To plot our calibrated spectrum against Doppler velocity, we use the following relation:
\begin{equation} \label{eq:velocity}
v = -c \frac{\Delta \nu}{\nu_0}
\end{equation}

Finally, to account for the times at which we observed, as well as to convert from galactic coordinates to topocentric coordinates (for which telescope pointing is more intuitive), we used the following rotation matrices:

Equatorial to galactic (radians):

\begin{equation}
\begin{bmatrix}
-.054876 & -.873437 & -.483835 \\
.494109 & -.444830 & .74698 \\
-.867666 & -.198706 & .455984
\end{bmatrix}
\end{equation}

Equatorial to hour-angle:

\begin{equation}
\begin{bmatrix}
\cos(\text{LST}) & \sin(\text{LST}) & 0 \\
\sin(\text{LST}) & -\cos(\text{LST}) & 0 \\
0 & 0 & 1
\end{bmatrix}
\end{equation}

Hour-angle to topocentric:

\begin{equation}
\begin{bmatrix}
-\sin(\text{latitude}) & 0 & \cos(\text{latitude}) \\
0 & -1 & 0 \\
\cos(\text{latitude}) & 0 & \sin(\text{latitude})
\end{bmatrix}
\end{equation}

To apply these matrices, we convert between angle pairs and rectangular vectors via

\begin{equation}
\vec{x} = 
\begin{bmatrix}
\cos(\phi) \cos(\theta) \\
\cos(\phi) \sin(\theta) \\
\sin(\phi)
\end{bmatrix}
\end{equation}
and
\begin{equation}
(\theta, \phi) = \text{atan2}(y, x), \sin^{-1}(z))
\end{equation}

For our second experiment, we seek to calculate to calculate the wavelength $\lambda_\text{sl}$ of a signal inside the waveguide based on the positions of the minima in the squared voltage, which we call nulls. We expect a linear dependence of null spacings on $\lambda_\text{sl}$. We use the following equation for this first experiment:

\begin{equation}
x_m = A + m \frac{\lambda_\text{sl}}{2}
\end{equation}

$m$ is the index of the null, $x_m$ is the position of null $m$, and $A$ represents a constant offset which can relate whether the far end of the waveguide is open or shorted.

The rectangular waveguide, to which we switch for the X-band phase of the experiment, abides a nonlinear relationship between the waveguide's width and the wavelengths of the input in the waveguide and in free space.

\begin{equation} \label{eq:xband}
\lambda_g = \frac{\lambda_\text{fs}}{[1 - (\frac{\lambda_\text{fs}}{2a})^2]^{1/2}}
\end{equation}

$\lambda_g$ is the guide wavelength, which we directly measure. $\lambda_\text{fs}$ is the free-space wavelength of the signal, which we will calculate using the known input frequency and the relation $c = f \lambda_\text{fs}$. $a$ is the width of the waveguide.

Finally, to analyze our results we define the reduced chi-squared function as follows:

\begin{equation}
\chi_r^2 = \frac{1}{N - M} \sum_i \frac{|y_i - \hat{y_i}|^2}{\sigma_i^2}
\end{equation}

Where $N$ is the number of data and $M$ is the number of fit parameters. For each measurement $i$, $y_i$ is the observed value, $\hat{y_i}$ is the value predicted by the fit, and $\sigma_i$ is the expected error on that measurement.

The reduced chi-squared function will provide something clear to minimize when fitting the data to a model. Specifically, we want to calculate the width $a$ based on our observations of null spacings, so we will calculate the value of $a$ which minimizes the reduced chi-squared.

\section{Methods}

\quad \quad However, these numbers changed when taking data for Cassiopeia--$(120^\circ, 0^\circ)$ in galactic coordinates, transformed to equatorial via the rotation matrices from the introduction--by which point our signal-to-noise ratio was too low. We switched to 10 dBm for both and proceeded with the pico-sampling at 62.5 / 6 MHz, the rate we used for all captures.

\section{Observations}

\quad \quad We collected data over many days of troubleshooting. The thermal data, as pictured in figure \ref{eq:thermal}, have a straightforward gap between them, but our gain calculation (of about 30) was well below that of other groups.

\begin{center}
 \begin{tabular}{||c c||} 
 \hline
 $f$ (GHz) & Null Positions (cm)\\ [0.5ex] 
 \hline
 7.0 & 8.8, 15.15 \\ 
 \hline
 7.2 & 11.95, 17.15 \\
 \hline
 7.5 & 10.05, 14.4 \\ 
 \hline
 8.0 & 10.15, 13.35, 16.7 \\
 \hline
 8.5 & 9.85, 12.7, 15.25 \\
 \hline 
 9.0 & 9.3, 12.3, 14.25, 16.7 \\ [1ex] 
 \hline
\end{tabular}
 \begin{tabular}{||c c||} 
 \hline
 $f$ (GHz) & Null Positions (cm)\\ [0.5ex] 
 \hline
 9.5 & 8.9, 11.15, 13.35, 15.45, 17.7 \\ 
 \hline
 10.0 & 8.45, 10.45, 12.3, 14.35, 16.5 \\
 \hline
 10.5 & 9.95, 11.75, 13.5, 15.45, 17.3 \\
 \hline
 11.0 & 9.35, 11.4, 12.65, 14.45, 16.2, 17.85 \\
 \hline
 11.5 & 8.95, 10.45, 12.0, 13.8, 15.15, 16.8 \\
 \hline
 12.0 & 8.4, 9.85, 11.35, 12.85, 14.4, 15.9, 17.45 \\ [1ex] 
 \hline
\end{tabular}
\end{center}

\section{Analysis}

\quad \quad Since the Horn was pointing straight up, we may say that the declination is equal to the latitude of the telescope. In our case, that is $37.873199^\circ \approx 0.661012$ radians. In the directly upward case, we also may say that the right ascension is equal to the local sidereal time (LST) of data-capture.

\begin{center}
 \begin{tabular}{||c c c c||} 
 \hline
 Label & Unix Time & Julian Date & LST \\ [0.5ex] 
 \hline
 On-line start & 1582937209.9447305 & 2458908.5325225084 & 0.8334479646673989 \\ 
 \hline
 On-line stop & 1582937704.4486437 & 2458908.5382459336 & 0.8695077621535888 \\
 \hline
 Off-line start & 1582937837.5090616 & 2458908.5397859844 & 0.8792106794915676 \\
 \hline
 Off-line stop & 1582938387.1761112 & 2458908.5461478718 & 0.9192930359644231 \\ [1ex] 
 \hline
\end{tabular}
\end{center}

\begin{center}
 \begin{tabular}{||c c c||} 
 \hline
 Label & PST & Doppler Correction (m / s)\\ [0.5ex] 
 \hline
 On-line start & 2/28/20 @ 16:46:49 & -27648.19266498681 \\ 
 \hline
 On-line stop & 2/28/20 @ 16:55:04 & -27897.548197380296 \\
 \hline
 Off-line start & 2/28/20 @ 16:57:17 & -27960.009947759394 \\
 \hline
 Off-line stop & 2/28/20 @ 17:06:27 & -28197.027228226034 \\ [1ex] 
 \hline
\end{tabular}
\end{center}

\section{Conclusions}

\quad \quad In conclusion, we obtained results strongly in favor of expected values for the HI line observation. Despite hiccups in the experimental design manifesting as noise spikes in our first pair of blocks, we picked out a likely candidate for the HI line there. The Cassiopeia observations were even more promising. We considered how fitting individual Gaussians to the lumps on fully calibrated power spectra can provide insight into the temperatures and relative velocities of different clouds of hydrogen, when we exchange the frequency axis for doppler velocities via equation \ref{eq:velocity}. We briefly probed some least-squares fits in a simple linear scenario (measuring the speed of light), questioning our assessments of error in the data, and failing to affirm the superiority of the least-squares approach. We highlighted the relevance of voltage standing wave ratio to impedance matching, and reiterated the theory behind the XBand data-fitting without success with actual data. 

\section{Acknowledgments}

\quad \quad I collected and re-collected data for the zenith and Cassiopeia spectra. Mehdi calculated the on and off frequencies for the local oscillator and helped me debug my data-taking code. Mehdi and Rebecca collected the XBand waveguide data, and I helped with the 3GHz data. Rebecca also helped to lay out data-collection schedules.

Theory and background provided by Aaron Parsons. ``LAB 2: Astronomy with the 21-cm Line and Waveguides.'' Updated February 2020.

\end{document}
